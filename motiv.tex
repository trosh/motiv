\documentclass[12pt,a4paper]{article}
\usepackage[margin=1.2in,top=1in,bottom=1in]{geometry}
\usepackage[french]{babel}
\usepackage[utf8]{inputenc}
\usepackage[T1]{fontenc}
\usepackage{hyperref}
\hypersetup{urlbordercolor={1 .5 0}}
\usepackage{newcent}
\selectfont
\usepackage{microtype}
\usepackage{graphicx}

\setlength{\parskip}{.2in}
\setlength{\parindent}{0pt}

\pagenumbering{gobble}

\begin{document}

\begin{minipage}{.5\textwidth}
\begin{flushleft}
John Gliksberg \\
14 rue Anatole France \\
92310 Sèvres \\
06 40 60 76 95 \\
\href{mailto:jg.trosh@gmail.com}
{jg.trosh@gmail.com}
\end{flushleft}
\end{minipage}

{
    \hspace*{\fill}
    \begin{minipage}{.5\textwidth}
    \begin{flushleft}
%    Groupe SAFRAN \\
%    Direction des Ressources Humaines \\
%    2 Boulevard du Général Martial Valin \\
%    75015 Paris \\[.07in]
%    \`A l’attention de Monsieur DELPIT \\[.07in]
    Le 6 février 2016
    \end{flushleft}
    \end{minipage}
}

\vspace*{\fill}

\begin{center}
    \large
    Recherche d'un stage M2 \\
    orienté calcul haute performance
\end{center}

\vspace*{\fill}

Madame, Monsieur,

Je recherche un stage orienté HPC de six mois entre mars et septembre,
dans le cadre d'un M2 en calcul haute performance.
Ce sera pour moi l'occasion de résoudre des problèmes scientifiques
et de découvrir la chaîne de production de Dassault Aviation.
Les metiers de l’aéronautique m’ont toujours fait rêver et
ce serait un grand plaisir de réunir ce domaine avec mon
champ d’études et un honneur d’effectuer ce stage au sein
d’une entreprise d’un tel renom.

Ma double licence maths physique m'offre
une base solide face aux questions scientifiques.
J'ai profité de cette période pour programmer de nombreux projets,
découvrir Unix, apprendre et approfondir de nombreux langages de programmation
et prendre part à plusieurs groupes de développeurs.
Mon choix de Master m'a permis de m'impliquer totalement dans ma passion,
afin d'en faire un projet professionnel tout en gardant
l'impulsion d'un parcours scientifique.

Bon codeur, j'apprends vite, aime maintenir des projets
soignés et m'intègre bien dans les équipes de travail.
Je m'attache à dépasser les attentes dans le but de chercher
la meilleure qualité de production de ma part et de mes collègues.
Je suis agile par nature et vais de l'avant pour conceptualiser un projet
dans son ensemble comme dans ses détails au sein d'un travail collectif.
Mon bilinguisme a été un atout culturel et linguistique à de nombreuses
reprises dans mon parcours informatique.

%L'été dernier, j'ai choisi de réaliser un stage optionnel de quatre mois
%chez Scilab Enterprises.
%En tant que développeur open source, j'ai fourni non seulement
%des codes bien documentés, des solutions concrètes pour l'industrialisation
%de produits scientifiques, mais aussi une implication dans les différents
%aspects de l'entreprise: j'ai travaillé avec le CTO pour restructurer
%et documenter le back-end de gestion des modules Scilab
%et ai participé à la communauté de développement pour packager et
%débugger des modules utilisateurs.
%Ce stage m'a beaucoup apporté et je veux mettre en œuvre mon expérience
%au sein de votre entreprise.

%Studieux et bien noté, je ne me contente pas de ma formation.
%Je continue de programmer pour moi-même et je participe à
%d'autres activités extracurriculaires: \newline
%Je suis tuteur en anglais pour des élèves handicapés;
%j'ai co-fondé une association pour promouvoir l'image de
%mon université à travers le sport, j'en ai réalisé le projet
%administratif et j'ai co-ordonné plus de trente membres actifs
%durant un an et demi.

%J'espère vous avoir montré mon implication et mes motivations.
Me tenant à votre disposition pour un prochain entretien,
je vous prie d'agréer, Madame, Monsieur,
mes sincères salutations.

\vspace*{\fill}

{
    \hspace*{\fill}
    \begin{minipage}{.4\textwidth}
        \includegraphics[width=.45\textwidth]{sig.png}
    \end{minipage}
}

\end{document}

