\documentclass[12pt,a4paper]{letter}
\usepackage[margin=1.2in]{geometry}
\usepackage[french]{babel}
\usepackage[utf8]{inputenc}

\begin{document}
\pagenumbering{gobble}
Bonjour,
\\[.3in]
Je recherche un stage en HPC
pour une période juin\---août.
Je recherche un poste où je pourrai
pleinement découvrir ma place dans
une chaine de production et appliquer
mes capacités à résoudre des
vrais problèmes.
\\[.2in]
J'ai initialement une formation en
maths et physique qui m'offre une
base solide face aux questions
scientifiques;
j'ai durant cette période passé mon
temps libre à programmer de nombreux
projets, découvrir unix et participer à des
communautés liées à la programmation.
Après avoir acquis ma double licence
j'ai choisi de rester à l'UVSQ et suivre la
branche d'informatique haute performance
et simulation pour m'impliquer à cent
pourcent dans ma passion pour
l'informatique tout en conservant
l'impulsion d'un parcours scientifique.
\\[.2in]
Je suis bon codeur, j'apprends vite, j'aime maintenir des projets soignés et je suis motivé pour m'intégrer dans une équipe. Pour moi, l'informatique c'est un plaisir personnel avant tout; pouvoir travailler avec ma source de plaisir c'est avoir la possibilité de mettre en oeuvre le meilleur de moi-même. Étant bilingue français\---anglais par mes parents et mon éducation internationale, je suis linguistiquement et culturellement un atout.
\\[.2in]
Studieux et bien noté, je ne me contente pas de ma formation. Je continue de programmer pour mon plaisir : je résous souvent des problèmes du projet Euler, j'aime reproduire des jeux, faire du data-bending et toutes sortes d'animations procédurales.
\\[.1in]
J'aime partager mes connaissances dans tous les domaines \---- je suis en particulier tuteur en anglais et je dispense en ce moment le cours d'une élève handicapée.
\\[.1in]
Je suis de plus représentant étudiant au conseil de l'UFR des Sciences de Versailles dans lequel je participe à la vie de l'université et découvre son fonctionnement interne.
\\[.1in]
Enfin j'ai participé à la création d'une association pour promouvoir l'image de mon université à travers le sport. J'ai pris part à la conception du projet, aux démarches administratives, à la mise en place des événements et j'ai monté le back-end informatique. C'était une occasion formidable de prendre un rôle décisionnel et surmonter les difficultés d'un projet en groupe.
\\[.2in]
J'espère vous avoir montré mon implication dans mon projet éducatif et professionnel et vous invite à voir dans mes nombreuses facettes la personne qu'il vout faut.
\\[.3in]
Avec l'expression de mes salutations,
\\[.2in]
John Gliksberg
\end{document}
